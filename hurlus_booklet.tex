%%%%%%%%%%%%%%%%%%%%%%%%%%%%%%%%%
% LaTeX model https://hurlus.fr %
%%%%%%%%%%%%%%%%%%%%%%%%%%%%%%%%%

% Needed before document class
\RequirePackage{pdftexcmds} % needed for tests expressions
\RequirePackage{fix-cm} % correct units

\documentclass[twoside]{book} % ,notitlepage
\usepackage[%
  papersize={105mm, 297mm},
  inner=12mm,
  outer=12mm,
  top=20mm,
  bottom=15mm,
  marginparsep=3pt,
  marginpar=7mm,
]{geometry}
\usepackage[fontsize=9.5pt]{scrextend} % for Roboto


\usepackage[dvipsnames]{xcolor}
\definecolor{rubric}{HTML}{000000} % the tonic

%%%%%%%%%%%%%%%%%
% Teinte macros %
%%%%%%%%%%%%%%%%%
\input{vendor/oeuvres/xsl/tei_latex/teinte}

%%%%%%%%%%%%%
% footnotes %
%%%%%%%%%%%%%
\renewcommand{\thefootnote}{\bfseries\textcolor{rubric}{\arabic{footnote}}} % color for footnote marks

%%%%%%%%%
% Fonts %
%%%%%%%%%
\usepackage[]{roboto} % SmallCaps, Regular is a bit bold
\setmainfont[
  ItalicFont={Roboto Light Italic},
]{Roboto}
\setsansfont{Roboto Light} % seen, if not set, problem with printer
\newfontfamily\fontrun[]{Roboto Condensed Light} % condensed runing heads
\renewcommand{\LettrineFontHook}{\bfseries\color{rubric}}
% \renewenvironment{labelblock}{\begin{center}\bfseries\color{rubric}}{\end{center}}

%%%%%%%%
% MISC %
%%%%%%%%

% Babel better
\usepackage[french]{babel}
\selectlanguage{french}

\newenvironment{quotebar}{%
    \def\FrameCommand{{\color{rubric!10!}\vrule width 0.5em} \hspace{0.9em}}%
    \def\OuterFrameSep{0pt} % séparateur vertical
    \MakeFramed {\advance\hsize-\width \FrameRestore}
  }%
  {%
    \endMakeFramed
  }
\renewenvironment{quoteblock}% may be used for ornaments
  {%
    \savenotes
    \setstretch{0.9}
    \begin{quotebar}
    \smallskip
  }
  {%
    \smallskip
    \end{quotebar}
    \spewnotes
  }


\renewcommand{\headrulewidth}{\arrayrulewidth}
\renewcommand{\headrule}{{\color{rubric}\hrule}}
\renewcommand{\lnatt}[1]{\marginpar{\sffamily\scriptsize #1}}

\titleformat{name=\chapter} % command
  [display] % shape
  {\vspace{1.5em}\centering} % format
  {} % label
  {0pt} % separator between n
  {}
[{\color{rubric}\huge\textbf{#1}}\bigskip] % after code
% \titlespacing{command}{left spacing}{before spacing}{after spacing}[right]
\titlespacing*{\chapter}{0pt}{-2em}{0pt}[0pt]

\titleformat{name=\section}
  [display]{}{}{}{}
  [\vbox{\color{rubric}\large\centering\textbf{#1}}]
\titlespacing{\section}{0pt}{0pt plus 4pt minus 2pt}{\baselineskip}

\titleformat{name=\subsection}
  [block]
  {}
  {} % \thesection
  {} % separator \arrayrulewidth
  {}
[\vbox{\large\textbf{#1}}]
% \titlespacing{\subsection}{0pt}{0pt plus 4pt minus 2pt}{\baselineskip}


\fancypagestyle{main}{%
  \fancyhf{}
  \setlength{\headheight}{1.5em}
  \fancyhead{} % reset head
  \fancyfoot{} % reset foot
  \fancyhead[RE]{\truncate{0.9\headwidth}{\fontrun\elbibl}} % book ref
  \fancyhead[LO]{\truncate{0.9\headwidth}{\fontrun\nouppercase\leftmark}} % Chapter title, \nouppercase needed
  \fancyhead[RO,LE]{\thepage}
}
\fancypagestyle{plain}{% apply to chapter
  \fancyhf{}% clear all header and footer fields
  \setlength{\headheight}{1.5em}
  \fancyhead[L]{\truncate{0.9\headwidth}{\fontrun\elbibl}}
  \fancyhead[R]{\thepage}
}

\newcommand\chapo{{%
  \vspace*{-3em}
  \centering\parindent0pt % no vskip ()

  {\LARGE\addfontfeature{LetterSpace=25}\bfseries \elauthorshort\par}\bigskip
  {\Large \eldate\par}\bigskip
  {\LARGE \eltitlefull}

  
  \bigskip
  {\color{rubric}\hline}
  \bigskip
  {\Large TEXTE LIBRE À PARTICIPATIONS LIBRES\par}
  \centerline{\small\color{rubric} {\href{https://hurlus.fr}{\dotuline{hurlus.fr}}}, tiré le \today}\par
  \bigskip
}}



\newcommand\cover{{%
  \thispagestyle{empty}
  \centering\parindent0pt

  {\LARGE\addfontfeature{LetterSpace=25}\bfseries \elauthorshort\par}\bigskip
  {\Large \eldate\par}\bigskip
  {\LARGE \eltitlefull}


  \vfill\null
  {\color{rubric}\setlength{\arrayrulewidth}{2pt}\hline}
  \vfill\null
  {\Large TEXTE LIBRE À PARTICIPATIONS LIBRES\par}
  \centerline{\href{https://hurlus.fr}{\dotuline{hurlus.fr}}, tiré le \today}\par
}}

\begin{document}
\pagestyle{empty}

\cover\newpage
\thispagestyle{empty}\hbox{}\newpage
\cover\newpage\noindent Les voyages de la brochure\par
\bigskip
\begin{tabularx}{\textwidth}{l|X|X}
  \textbf{Date} & \textbf{Lieu}& \textbf{Nom/pseudo} \\ \hline
  \rule{0pt}{25cm} &  &   \\
\end{tabularx}
\newpage
\addtocounter{page}{-4}


\thispagestyle{empty}
\chapo

{\it\elabstract}
\bigskip
\makeatletter\@starttoc{toc}\makeatother % toc without new page
\bigskip

\pagestyle{main} % after style
\setcounter{footnote}{0}
\setcounter{footnoteA}{0}

%text%


\pagestyle{empty}
\clearpage
% 2 empty pages maybe needed for 4e cover
\ifnum\modulo{\value{page}}{4}=0 \hbox{}\newpage\hbox{}\newpage\fi
\ifnum\modulo{\value{page}}{4}=1 \hbox{}\newpage\hbox{}\newpage\fi


\hbox{}\newpage
\ifodd\value{page}\hbox{}\newpage\fi
{\centering\color{rubric}\bfseries\noindent\large
  Hurlus ? Qu’est-ce.\par
  \bigskip
}
\noindent Des bouquinistes électroniques, pour du texte libre à participations libres,
téléchargeable gratuitement sur \href{https://hurlus.fr}{\dotuline{hurlus.fr}}.\par
\bigskip
\noindent Cette brochure a été produite par des éditeurs bénévoles.
Elle n’est pas faite pour être possédée, mais pour être lue, et puis donnée, ou déposée dans une boîte à livres.
En page de garde, on peut ajouter une date, un lieu, un nom ;
comme une fiche de bibliothèque en papier qui enregistre \emph{les voyages de la brochure}.
\par

Ce texte a été choisi parce qu’une personne l’a aimé,
ou haï, elle a pensé qu’il partipait à la formation de notre présent ;
sans le souci de plaire, vendre, ou militer pour une cause.
\par

L’édition électronique est soigneuse, tant sur la technique
que sur l’établissement du texte ; mais sans aucune prétention scolaire, au contraire.
Le but est de s’adresser à tous, sans distinction de science ou de diplôme.
\par

Cet exemplaire en papier a été tiré sur une imprimante personnelle
  ou une photocopieuse. Tout le monde peut le faire.
Il suffit de
télécharger un fichier sur \href{https://hurlus.fr}{\dotuline{hurlus.fr}},
d’imprimer, et agrafer (puis lire et donner).\par

\bigskip

\noindent PS : Les hurlus furent aussi des rebelles protestants qui cassaient les statues dans les églises catholiques. En 1566 démarra la révolte des gueux dans le pays de Lille. L’insurrection enflamma la région jusqu’à Anvers où les gueux de mer bloquèrent les bateaux espagnols.
Ce fut une rare guerre de libération dont naquit un pays toujours libre : les Pays-Bas.
En plat pays francophone, par contre, restèrent des bandes de huguenots, les hurlus, progressivement réprimés par la très catholique Espagne.
Cette mémoire d’une défaite est éteinte, rallumons-la. Sortons les livres du culte universitaire, débusquons les idoles de l’époque, pour les démonter.

\end{document}
